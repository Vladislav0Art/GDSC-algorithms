\Subsection{Master Theorem}

Algorithms that are written in a recursive manner oftentimes utilize \textit{divide-and-conquer} technique which implies devision of the task into smaller subtasks that are processed by further recusive calls of the algorithm; once the subtask is small enough it is considered as a base case and processed manually. Some examples include \textbf{Merge sort algorithm}, \textbf{Binary search tree traversal}, etc.

For such algorithms we need to define their asymptotics. \textbf{Master Theorem} is a generalized method that yields asymptotically tight bounds for divide and conquer algorithms [\href{https://en.wikipedia.org/wiki/Master_theorem_(analysis_of_algorithms)#Generic_form}{wiki}].

\begin{theorem}
    \textbf{Master Theorem}

    Consider the following recurrence relation: $T(n) = a \cdot T(\frac{n}{b}) + f(n)$ where $f(n) = n^c$ for constants $a > 0, b > 1, c \geq 0$; let $k = \log_{b}{n}$ be the recursion depth. Then the following holds:

    \begin{equation}
        \begin{cases}
            T(n) = \Theta(a^k) = \Theta(n^{\log_{b}{a}}), \quad a > b^c\\
            T(n) = \Theta(f(n)) = \Theta(n^c), \quad a < b^c\\
            T(n) = \Theta(k \cdot f(n)) = \Theta(n^c \cdot \log{n}), \quad a = b^c\\
        \end{cases}
    \end{equation}

\end{theorem}



\begin{proof}

    $$ T(n) = f(n) + a \cdot T(\frac{n}{b}) = f(n) + a f(\frac{n}{b}) + a^2 f(\frac{n}{b^2}) + ... + a^k f(\frac{n}{b^k}) \quad | \ f(n)=n^c$$
    $$ T(n) = n^c + a\cdot (\frac{n}{b})^c + a^2\cdot (\frac{n}{b^2})^c + ... + a^k\cdot (\frac{n}{b^k})^c $$
    $$ T(b) = n^c \cdot (1 + \frac{a}{b^c} + (\frac{a}{b^c})^2 + ... + (\frac{a}{b^c})^k) $$

    Let $q = \frac{a}{b^c}$ and $S(q) = 1 + q + ... + q^k$:

    1. If $q=1$: $S(q) = 1 + 1 + ... + 1 = k+1 = \log_{b}{n} + 1 \implies T(n) = \Theta(f(n) \cdot \log{n})$.

    2. If $q<1$: $S(q)$ is a geometric progression, thus it is equal to $S(q) = \frac{1-q^{k+1}}{1-q} = const = \Theta(1) \implies T(n) = \Theta(f(n))$.

    3. If $q>1$: $S(q) = q^k + \frac{q^k-1}{q-1} = \Theta(q^k) \implies T(n) = \Theta(a^k \cdot (\frac{n}{b^k})^c) = \Theta(a^k)$.

    \textbf{Note}: $f(n)$ could be $O(n^c)$; it does not violate the proof ($f(n) = O(n^c) = C \cdot n^c$).

\end{proof}


\Subsection{Generalized Master Theorem}

\begin{theorem}
    \textbf{Generalized Master Theorem}

    In the case of $f(n) = n^c \cdot \log_{d}{n}$ Master Theorem still holds:

    $T(n) = a \cdot T(\frac{n}{b}) + n^c \cdot \log_{d}{n}$, $a>0, \ b>1, \ c\geq 0, \ d \geq 0$.

    \begin{equation}
        \begin{cases}
            T(n) = \Theta(a^k) = \Theta(n^{\log_{b}{a}}), \quad a > b^c\\
            T(n) = \Theta(f(n)) = \Theta(n^c \cdot \log^{d}{n}), \quad a < b^c\\
            T(n) = \Theta(k \cdot f(n)) = \Theta(n^c \cdot \log^{d+1}{n}), \quad a = b^c\\
        \end{cases}
    \end{equation}

\end{theorem}


\Subsection{Algorithm for recurrence relations}

There are also recurrence relations with the following form:

$T(n) = a_0 \cdot T(n - p_0) + a_1 \cdot T(n - p_1) + ... + a_k \cdot T(n-p_k) \quad a_i, p_i > 0, \ \sum p_i > 1$

There exists an algorithm of how to find asymptotics for such relations:

\begin{theorem}
    \textbf{Algorithm for recurrence relations}

    Given $T(n)$ of the above form with the above constants, then the following holds:

    $T(n) = \Theta(\alpha^n)$, such that $\alpha > 1$ and it is the \textbf{only root} of the equation: $\alpha^n = a_0 \cdot \alpha^{n-p_0} + ... + a_k \cdot \alpha^{n-p_k}$

    % TODO: add proof

\end{theorem}


\begin{example}
    \textbf{Use of Master Theorem}

    1. $T(n) = 4 \cdot T(\frac{n}{2}) + 20 \cdot n^\frac{3}{2}$

    $a=4, \ b=2, \ c=\frac{3}{2}, \ f(n)=20\cdot n^\frac{3}{2} \implies a=4 > b^c=\sqrt{8} \implies T(n) = \Theta(n^{log_{b}{a}}) = \Theta(n^2)$.

    2. Merge sort algorithm recurrence relation: $T(n) = 2 \cdot T(\frac{n}{2}) + C\cdot n^1$

    $a=2, \ b=2, \ c=1 \implies a=4 = b^c=2^1 \implies T(n) = \Theta(n^1 \cdot \log{n})$

\end{example}


\begin{example}
    \textbf{Use of Algorithm for recurrence relations}

    1. $T(n) = T(n-1) + 6 \cdot T(n-2)$

    $T(n) = \Theta(\alpha)$, notice that $\alpha = 3$ satisfies the equation: $3^n = 3^{n-1} + 6 \cdot 3^{n-2}$.

    2. $T(n) = T(n-1) + T(n-2) + T(n-3)$

    $1 = \alpha^{-1} + \alpha^{-2} + \alpha^{-3} \implies \alpha \approx 1.839$

\end{example}