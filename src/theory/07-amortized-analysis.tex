\Subsection{Definition and general perception of the concept}

Remember that we were talking about \textbf{std::vector} we said that its \textbf{push\_back} operation works for an average of $\Theta(1)$. It was due to presence of 2 \textbf{distinct states}:

1. vector has enough capacity to fit the next pushed element.

2. vector does not have enough capacity and has to make itself twice bigger (i.e. reallocating a memory buffer of size $2\cdot N$).

There are definetely more complicated senarios where number of such \textit{interesting states} is much greater, thus we need a unified approach of how to define this average, or \textbf{amortized}, time for an operation.

\begin{definition}
    \textbf{The amortized analysis}

    The amortized analysis is an approach that allows to determine an average running time (time complexity) of operations $o_1$, $o_2$, ..., $o_k$ in a sequence $S$ over that sequence $S$.

\end{definition}

There are several methods that are referred to as amortized analysis (\href{https://en.wikipedia.org/wiki/Amortized_analysis}{wiki}). We are going to discuss \textbf{Potential method}.

\Subsection{Amortized analysis: Potential method}

\begin{definition}
    \textbf{Potential method}

    % TODO: use \varphi + use \mathbb instead of \mathbf, inf -> \infty

    Introduce a \textbf{potential function} called $\Phi: \mathbf{S} \to \mathbf{R_{0}^{+}}$ where $\mathbf{S}$ is a set of states of the considered data structure and $R_{0}^{+} = [0, +\inf)$, i.e. the potential function maps states of the data structure to some non-negavite values. As an important edge case for the initial state $S_{init}$: $\Phi(S_{init}) = 0$.

    Let $o_i$ be an individual operation within some sequence of operations named $Q$. Let $S_{i-1}$ be the state of the considered data structure before the execution of the operation $o_i$ and $S_{i}$ be the state after the execution of $o_i$. Let $\Phi$ be a chosen potential function, then the amortized time for an operation $o_i$ is defined as follows:

    $ T_{a}(o_i) = T_{r}(o_i) + \left (\Phi(S_{i}) - \Phi(S_{i-1}) \right ) $ where:

    1. $T_{a}(o)$ - amortized time of the operation.

    2. $T_{r}(o)$ - real/actual time spent on the operation.
\end{definition}

\begin{theorem}
    \textbf{Potential method yields an upper bound}

    The amortized time of a sequence of operations always yields an \textbf{upper bound} of the the real/actual time for the considered sequence of operations, i.e.:

    $\forall O = o_1, o_2, ..., o_n$ be a sequence of operations, define:

    1. $T_{a}(O) = \Sigma_{i=1}^{n} T_{a}(o_i)$ - amortized time of the sequence $O$

    2. $T_{r}(O) = \Sigma_{i=1}^{n} T_r(o_i)$ - real time of the sequence $O$

    Then: $T_r(O) \leq T_a(O)$

\end{theorem}

\begin{proof}

    As definition for $T_a(O)$ suggests:

    $$ T_{a}(O) = \Sigma_{i=1}^{n} (T_{r}(o_i) + \Phi(S_i) - \Phi(S_{i-1})) = T_{r}(O) + \Phi(S_n) - \Phi(S_0) $$
    $$ T_{a}(O) = T_{r}(O) + \underbrace{\Phi(S_n)}_{\cdot \geq 0} - \underbrace{\Phi(S_0)}_{\cdot = 0} \geq T_{r}(O) $$

\end{proof}

\underline{Note}: Generally speaking, $\Phi$ could be any function that satisfies the constraints but the idea is to find such $\Phi$ that would represent the closest upper bound of the real/actual time for the considered sequence of operations.

\begin{example}
    \textbf{Amortized analysis for std::vector::push\_back operation}

    1. Analyse push\_back operation that expands the vector:

    Let $\Phi = 2 \cdot s - N$ where the $s$ is the actual size of a vector and $N$ is its capacity. Remember that the expansion occurrs once $s == N$.

    Let's consider a push\_back operation that doubles the size of the vector:

    1. $T_{r} = s+1 = N+1$ - because we need to copy all the existing elements in a new buffer + place a new element; here $s=N$ because the extention could only happen once $s=N$.

    2. $\Phi_0 = 2\cdot s - N \geq 0$ - value of the potential function before the operation.

    3. $\Phi_1 = 2\cdot (s + 1) - 2N \geq 0$ - value of the potential function after the operation.

    Thus, we have:

    $T_a = T_r + \Phi_1 - \Phi_0 = N+1 + (2s + 2 - 2N - 2s + N) = N+1 + (2-N) = N-N+3 = \Theta(1)$.\newline


    2. Analyse push\_back operation that does not expand the vector:

    Notice that in this case $\Delta\Phi = \Phi_1 - \Phi_0 = const$ and $T_r = const$, thus $T_a = const = \Theta(1)$.\newline

    For more examples check the \href{https://en.wikipedia.org/wiki/Potential_method#Multi-Pop_Stack}{wiki} page.


\end{example}