\Subsection{Hash table with Sliding Window}

\begin{problem}

    Given an integer array $a$ and an integer $k$, return the number of \textbf{good subarrays} of $a$.

    A \textbf{good array} is an array where the number of different (distinct) integers in that array is exactly $k$. A subarray is a \textbf{contiguous part} of an array.

    For example, $[1, 2, 3, 1, 2]$ has $3$ different integers: $1, 2, 3$.

    Asymptotics: $O(n)$ in time and space. \newline


    \underline{Note}: you can solve this problem on LeetCode here \href{https://leetcode.com/problems/subarrays-with-k-different-integers/description/}{992. Subarrays with K Different Integers}.

\end{problem}

\begin{example}

    1. $a = [1,2,1,2,3]$, $k = 2$, the answer will be $7$, the good subarrays are:

    $[1,2], [2,1], [1,2], [2,3], [1,2,1], [2,1,2], [1,2,1,2]$.\newline

    2. $a = [1,2,1,3,4]$, $k = 3$, the answer will be $3$, the good subarrays are:

    $[1,2,1,3], [2,1,3], [1,3,4]$.

\end{example}



\begin{solution}

Idea: $exactly(k) = atMost(k) - atMost(k-1)$, thus let's count the number of subarrays with at most $k$ ($\leq k$) distinct numbers and the number of subarrays with at most $k-1$ distinct numbers.

In order to count distinct numbers if a sliding window we need keep the count of each distinct value in the sliding window.

Once the right border of the window goes forward we increment the count of a new value $a_j$, and if it the count became $1$, then increment $cnt$ (since the number of distinct numbers increased by one).

Once the left border of the window goes forward we decrement the count of a tracked value $a_i$ (before incrementing $i$) and if this value becomes $0$, we decrement the number of distinct values.

Let's keep a sliding window $[i, j]$ that will contain $\leq k$ distinct elements:

1. If current number of distinct values is $\leq k$ we only need to move forward the right boundary $j$ and on each iteration increase the result by $(j - i + 1)$ because the number of subarrays that end in $j$ and contain $\leq k$ distinct values is the number of suffixes the subarray $a[i..j]$ has (suffixes are: $[i+1..j]$, $[i+2..j]$, ..., $[j-1..j]$, $[j..j]$). This way we will count all the subarrays of $a$ that contain $\leq k$ elements.

2. Once the number of distinct values $cnt$ inside the sliding window has become $(k+1)$ we need to move forward the left boundary $i$ until $cnt$ becomes equal to $k$.

By the above algorithm we found the result of the function $atMost(k)$, now do the same for $atMost(k-1)$ and return the their difference.


\begin{lstlisting}[language=C++]
int atMost(vector<int>& nums, int k) {
    unordered_map<int, int> counts;
    int i = 0;
    int cnt = 0;
    int result = 0;

    for (int j = 0; j < nums.size(); ++j) {
        if (counts[nums[j]]++ == 0) {
            ++cnt;
        }

        if (cnt <= k) {
            result += (j - i + 1);
        }

        while (i <= j && cnt > k) {
            --counts[nums[i]];
            if (counts[nums[i]] == 0) {
                --cnt;
            }
            ++i;

            if (cnt == k) {
                result += (j - i + 1);
            }
        }
    }

    return result;
}

int exact(vector<int>& nums, int k) {
    return atMost(k) - atMost(k-1);
}
\end{lstlisting}

\end{solution}