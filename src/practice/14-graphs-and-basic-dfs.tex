\begin{problem}\textbf{Cycle in Graph}

    Given a graph $g$ of $n$ verticies and $m$ edges. Find any cycle in the given graph and print its verticies.
\end{problem}


\begin{solution}

    The idea is to implement dfs with \textbf{two colors}:

    1. \textit{Color 1} denotes that this vertex was visited but yet not left by the dfs.

    2. \textit{Color 2} denotes that this vertex was left by the dfs.

    The cycle is a situation in the dfs algorithm when it encounters any vertex that has a color $1$. Once such vertex is found the graph has a cycle.

    In order to print its verticies we need to keep track of the visited but yet not left verticies (i.e. verticies of color 1). We can implement it by inserting a currently considered vertex inside an vector and remove it from the vector once all its neighbouring verticies are traversed. Once the cycle is found ending at the vertex $v$ of color $1$, we will go through the elements of this tracking vector until and print its elements until $v$ is found.

    \begin{lstlisting}[language=C++]
    bool dfs(int v, vector<int>& trace) {
        if (marked[v] == 2) return false;
        if (marked[v] == 1) {
            // found cycle
            std::cout << v << " ";
            while(trace.back() != v) {
                std::cout << trace.back() << " ";
                trace.pop_back();
            }
            return true;
        }

        marked[v] = 1;
        trace.push_back(v);

        // traverse neighbouring verticies
        bool found = false;
        for (int u : graph[v]) {
            found = dfs(u);
            if (found) break;
        }

        trace.pop_back();
        return found;
    }
    \end{lstlisting}

\end{solution}