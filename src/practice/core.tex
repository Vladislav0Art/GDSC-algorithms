\documentclass[a4paper,12pt]{article}
% \usepackage[T2A]{fontenc}
\usepackage{lmodern}
\usepackage[utf8]{inputenc}
\usepackage[T1]{fontenc}
\usepackage[english]{babel}
\usepackage[usenames,dvipsnames]{xcolor}
\usepackage{xspace}
\usepackage{mathrsfs}
\usepackage{graphicx}
\usepackage{textcomp}

\pagestyle{plain}
\usepackage[
  left=0.50in,
  right=0.50in,
  top=0.8in,
  bottom=0.7in,
  headheight=0.8in]{geometry}
\pagenumbering{gobble}

\setlength{\parskip}{0.15cm}

\usepackage{indentfirst}

\usepackage{hyperref}
\hypersetup{
  colorlinks,
  citecolor=black,
  filecolor=black,
  linkcolor=blue,
  urlcolor=blue
}

\usepackage{paralist}
\usepackage{cancel}
\usepackage{textcomp}
\usepackage{gensymb}
\usepackage{mdframed}
\usepackage{lastpage}
\usepackage{microtype}
\usepackage[super]{cite}
\usepackage{fancyhdr}
\pagestyle{fancy}


\newcommand\Section[2]{
  \newpage % new page
  \stepcounter{section}
  \bigskip
  \phantomsection
  \addcontentsline{toc}{section}{\arabic{section}. #1}
  \begin{center}
    {\huge \bf \arabic{section}. #1}\\
  \end{center}
  \bigskip
  \gdef\SectionName{#1}
  \gdef\AuthorName{#2}

  \lhead{\ShortCourseName}
  \chead{}
  \rhead{\SectionName}

  \cfoot{
%    \topskip0pt\vspace*{\fill}
    \thepage~from~\pageref*{LastPage}
%    \vspace*{\fill}
  }

  \renewcommand{\headrulewidth}{0.15 mm}

  \ifx\LaconicFooter\undefined
  \lfoot{
%    \topskip0pt\vspace*{\fill}
    Chapter \texttt{\#\arabic{section}}
%    \vspace*{\fill}
  }
  \rfoot{
%    \topskip0pt\vspace*{\fill}
    Author: \AuthorName
%    \vspace*{\fill}
  }
  \renewcommand{\footrulewidth}{0.15 mm}
  \fi
}



\newcommand\Subsection[1]{
  \subsection{#1}
}

\newcommand\Subsubsection[1]{
  \subsubsection{#1}
}

\newcommand\slashnl{~\\*[-26pt]}
\newcommand\slashn{~\\*[-22pt]}
\newcommand\slashns{~\\*[-18pt]}
\newcommand\slashnss{~\\*[-14pt]}
\newcommand\slashnsss{~\\*[-10pt]}

\newcommand{\makegood}{
  \ifx\ShortCourseName\undefined
  \gdef\ShortCourseName{\CourseName}
  \fi


  \ifx\NoTitlePage\undefined

  \ifx\CustomTitle\undefined
    \title{\CourseName}
    \maketitle
  \else
    \pagestyle{empty}
    \CustomTitle
  \fi
  \tableofcontents
  \pagebreak
  \fi

  \pagestyle{fancy}
  \pagenumbering{arabic}
  \setcounter{page}{1}

  \lhead{\ShortCourseName}

  \ifdefined\ENABLEDMATH
  \renewcommand\proofname{\em\textbf{Proof}}
  \else
  \fi
}


\newcommand{\TODO}[1][]{
  \vspace{0.2em}
  \textbf{{\bf\color{red} TODO:} #1}
  \vspace{0.2em}
}


\newcommand\enablemath{
  \usepackage{amsmath,amsthm,amssymb,mathtext}
  \usepackage{thmtools}
  \usepackage{tikz}

  \newcommand\R{\ensuremath{\mathbb{R}}\xspace}
  \newcommand\Q{\ensuremath{\mathbb{Q}}\xspace}
  \newcommand\N{\ensuremath{\mathbb{N}}\xspace}
  \newcommand\Z{\ensuremath{\mathbb{Z}}\xspace}
  \newcommand\CC{\ensuremath{\mathbb{C}}\xspace}

  \DeclareRobustCommand{\divby}{%
    \mathrel{\vbox{\baselineskip.65ex\lineskiplimit0pt\hbox{.}\hbox{.}\hbox{.}}}%
  }
  \newcommand\notmid{\centernot\mid}

  \let\Im\relax
  \let\Re\relax
  \DeclareMathOperator\Im{Im}
  \DeclareMathOperator\Re{Re}
  \DeclareMathOperator\res{res}

  \DeclareMathOperator*{\lcm}{lcm}
  \newcommand\vphi{\varphi}

  \usepackage{
    nameref,
    hyperref,
    cleveref}

  \ifx\ThmSpacing\undefined
  \def\ThmSpacing{9pt}
  \fi

  \ifx\ThmNamespace\undefined
  \def\ThmNamespace{section}
  \fi

  \declaretheoremstyle[
    spaceabove=\ThmSpacing, spacebelow=\ThmSpacing,
    headfont=\slshape\bfseries,
    bodyfont=\normalfont,
    postheadspace=0.5em,
  ]{thmstyle_def}

  \declaretheoremstyle[
    spaceabove=\ThmSpacing, spacebelow=\ThmSpacing,
    postheadspace=0.5em,
  ]{thmstyle_thm}

  \declaretheoremstyle[
    spaceabove=\ThmSpacing, spacebelow=\ThmSpacing,
    headfont=\itshape\bfseries,
    notefont=\itshape\bfseries, notebraces={}{},
    bodyfont=\normalfont,
    postheadspace=0.5em,
  ]{thmstyle_cons}

  \declaretheoremstyle[
    spaceabove=\ThmSpacing, spacebelow=\ThmSpacing,
    headfont=\bfseries,
    notefont=\bfseries, notebraces={}{},
    bodyfont=\normalfont,
    postheadspace=0.5em,
  ]{thmstyle_examp}

  \declaretheoremstyle[
    spaceabove=\ThmSpacing, spacebelow=\ThmSpacing,
    headfont=\ttfamily\itshape,
    notefont=\ttfamily\itshape, notebraces={}{},
    bodyfont=\normalfont,
    postheadspace=0.5em,
  ]{thmstyle_remark}

  \declaretheorem[numberwithin=\ThmNamespace, name=Theorem, style=thmstyle_thm]{theorem}
  \declaretheorem[numberwithin=\ThmNamespace, name=Definition, style=thmstyle_def]{definition}
  \declaretheorem[sibling=theorem, name=statement, style=thmstyle_thm]{statement}
  \declaretheorem[numbered=no, name=Note, style=thmstyle_remark]{remark}
  \declaretheorem[numbered=no, name=Lemma, style=thmstyle_thm]{lemma}
  \declaretheorem[numbered=no, name=Implication, style=thmstyle_cons]{consequence}
  \declaretheorem[numbered=no, name=Example, style=thmstyle_examp]{example}
  \declaretheorem[numbered=no, name=Properties, style=thmstyle_cons]{properties}
  \declaretheorem[numbered=no, name=Properties, style=thmstyle_cons]{property}
  \declaretheorem[numbered=no, name=Problem, style=thmstyle_cons]{problem}
  \declaretheorem[numbered=no, name=Solution, style=thmstyle_cons]{solution}


  \newcommand\thmslashn{\slashn}

  % Examples:
  % \begin{theorem} theorem-statement \end{theorem}
  %
  % Use theorem name in a title:
  % \begin{definition}[My name] the definition \end{definition}
  %
  % Create a mark for later use of a link:
  % \begin{statement}\label{stm:identifier} the statement \end{statement}
  %
  % \begin{statement}\label{otherlabel}
  %   \hyperref[stm:identifier]{Theorem}.
  %   \ref{stm:identifier}
  %   \autoref{stm:identifier} ‘‘\nameref{stm:identifier}’’,
  % \end{statement}

  \newcommand\eps{\varepsilon}
  \renewcommand\le{\leqslant}
  \renewcommand\ge{\geqslant}
  \newcommand\empysetold{\emptyset}
  \renewcommand\emptyset{\varnothing}

  \newcommand\ENABLEDMATH{YES}
}

\newcommand\enablecode{
  \usepackage{listings}
  \lstset{
    belowcaptionskip=1\baselineskip,
    breaklines=true,
    frame=L,
    xleftmargin=\parindent,
    showstringspaces=false,
    basicstyle=\footnotesize\ttfamily,
    keywordstyle=\bfseries\color{blue},
    commentstyle=\itshape\color{Maroon},
    identifierstyle=\color{black},
    stringstyle=\color{orange},
    numbers=left
  }

  \lstdefinestyle{supercpp} {
    language=C++,
    deletekeywords={int, long, char, short, unsigned, signed,
      uint64\_t, int64\_t, uint32\_t, int32\_t, uint16\_t, int16\_t, uint8\_t, int8\_t,
      size\_t, ptrdiff\_t, \#include,\#define,\#if,\#ifdef,\#ifndef},
    classoffset=1,
    morekeywords={vector,stack,queue,set,map,unordered\_set,unordered\_map,deque,array,string,multiset,multimap,
      int, long, char, short, unsigned, signed,
      uint64\_t, int64\_t, uint32\_t, int32\_t, uint16\_t, int16\_t, uint8\_t, int8\_t,
      size\_t, ptrdiff\_t
    },
    keywordstyle=\bfseries\color{green!40!black},
    classoffset=0,
    classoffset=2,
    morekeywords={std},
    keywordstyle=\bfseries\color{ForestGreen},
    classoffset=0,
    more comment=[l][\bfseries\color{purple!99!black}]{\#}
  }
}